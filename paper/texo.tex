\documentclass[nofonts]{ctexart}

% \defaultfontfeatures{Mapping=tex-text}

\setCJKmainfont{Hiragino Sans GB}
\setmainfont{Myriad Pro}

\title{Texo -- 一个文本排版引擎的设计与实现}
\author{Jjgod Jiang <gzjjgod@gmail.com>}

\begin{document}

\maketitle

\begin{abstract}
iPhone OS 缺少一个轻量级的文本排版引擎,而 Mac OS X 本身提供的
Core Text 引擎不能够在其他平台下使用,Texo 开发的目的是尽可能
利用这两个系统共有的基础框架,开发一个尽可能高效的轻量级文本排
版引擎,提供较为复杂的富文本排版支持。
\end{abstract}

在 iPhone OS 下阅读大量的文本内容遇到的瓶颈是文本的载入效率和
排版效率问题,现有的文本阅读器往往使用系统自带的 WebKit 引擎提
供复杂文本的布局与显示,然而 WebKit 为了支持超文本复杂的效果,
引入了许多额外开销,在阅读纯文本内容时这些开销本不必存在。

\end{document}

